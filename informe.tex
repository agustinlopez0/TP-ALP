\documentclass{article}
\usepackage{graphicx} % Required for inserting images
\usepackage{amssymb}
\usepackage{proof} % Para \infer

\usepackage{amsthm}

\newtheorem{proposition}{Proposición}
\usepackage{adjustbox}
\usepackage{syntax}
\usepackage{amsthm}
\usepackage{amsmath}
\renewcommand{\syntleft}{}
\renewcommand{\syntright}{}
% TikZ y la librería que permite "zigzag"
\usepackage{tikz}
\usetikzlibrary{decorations.pathmorphing}

% Comando: \zigzagarrow[<longitud>]  (longitud por defecto 0.95cm)
\newcommand{\zigzagarrow}[1][0.95cm]{%
  \mathrel{%
    \tikz[baseline=-0.6ex]{%
      \draw[->, line width=0.6pt, decorate,
            decoration={zigzag,segment length=3.5pt,amplitude=1.1pt}]
            (0,0) -- (#1,0);
    }%
  }%
}

\begin{document}

\section*{Ejercicio 1}
\subsection*{Sintaxis abstracta}
\begin{grammar}

<intexp> ::= <nat> 
        \alt <var> 
        \alt <var> \texttt{++} 
        \alt \texttt{$-_{u}$} <intexp> 
        \alt <intexp> \texttt{+} <intexp>
        \alt <intexp> \texttt{$-_{b}$} <intexp>
        \alt <intexp> $\,\times\,$ <intexp>
        \alt <intexp> $\,\div\,$ <intexp>
        %\alt \texttt{(} <intexp> \texttt{)}
        
<boolexp> ::= \textbf{\texttt{true}}
        \alt \textbf{\texttt{false}}
        \alt <intexp> $\,=\,$ <intexp>
        \alt <intexp> $\,\neq\,$ <intexp>
        \alt <intexp> \texttt{\textless} <intexp>
        \alt <intexp> \texttt{\textgreater} <intexp>
        \alt <boolexp> \(\land\) <boolexp>
        \alt <boolexp> \(\lor\) <boolexp>
        \alt \(\lnot\) <boolexp>
        %\alt \texttt{(} <boolexp> \texttt{)}


<comm> ::= \textbf{\texttt{skip}}
        \alt <var> \texttt{=} <intexp>
        \alt <comm> \texttt{;} <comm>
        %\alt \textbf{\texttt{if}} <boolexp> \textbf{\texttt{then}} <comm> 
        \alt \textbf{\texttt{if}} <boolexp> \textbf{\texttt{then}} <comm> \textbf{\texttt{else}} <comm> 
        \alt \textbf{\texttt{repeat}} <comm> \textbf{\texttt{until}} <boolexp>
        %\alt \textbf{\texttt{case}} <casexp>


%<casexp> ::= <boolexp> <comm> <casexp>
%       \alt $\epsilon$
\end{grammar}

\subsection*{Sintaxis concreta}
\begin{grammar}

<digit> ::= \texttt{'0'} \alt \texttt{'1'} \alt \dots \alt \texttt{'9'}

<letter> ::= \texttt{'a'} \alt \dots \alt \texttt{'Z'}

<nat> ::= <digit> \alt <digit> <nat>

<var> ::= <letter> \alt <letter> <var>

<intexp> ::= <nat> 
        \alt <var> 
        \alt <var> \texttt{++} 
        \alt \texttt{'-'} <intexp> 
        \alt <intexp> \texttt{'+'} <intexp>
        \alt <intexp> \texttt{'-'} <intexp>
        \alt <intexp> \texttt{'*'} <intexp>
        \alt <intexp> \texttt{'/} <intexp>
        \alt \texttt{'('} <intexp> \texttt{')'}
        
<boolexp> ::= \textbf{\texttt{'true'}}
        \alt \textbf{\texttt{'false'}}
        \alt <intexp> \texttt{'=='} <intexp>
        \alt <intexp> \texttt{'!='} <intexp>
        \alt <intexp> \texttt{'\textless'} <intexp>
        \alt <intexp> \texttt{'\textgreater'} <intexp>
        \alt <boolexp> \texttt{'\&\&'} <boolexp>
        \alt <boolexp> \texttt{'||'} <boolexp>
        \alt '\(!\)' <boolexp>
        \alt \texttt{'('} <boolexp> \texttt{')'}


<comm> ::= \textbf{\texttt{'skip'}}
        \alt <var> \texttt{'='} <intexp>
        \alt <comm> \texttt{';'} <comm>
        \alt \textbf{\texttt{'if}'} <boolexp> \texttt{'\{'} <comm> \texttt{'\}'} 
        \alt \textbf{\texttt{'if}'} <boolexp> \texttt{'\{'} <comm> \texttt{'\}'} 
        \textbf{\texttt{'else'}} \texttt{'\{'} <comm> \texttt{'\}'} 
        \alt \textbf{\texttt{'repeat'}} \texttt{'\{'} <comm> \texttt{'\}'}  \textbf{\texttt{'until'}} <boolexp>
        \alt \textbf{\texttt{'case'}} <casexp>

<casexp> ::= <boolexp> \textbf{\texttt{':'}} \texttt{'\{'} <comm> \texttt{'\}'} <casexp>
       \alt $\epsilon$

\end{grammar}

\section*{Ejercicio 4}
\begin{adjustbox}{center,max width=\textwidth}
    \infer[VARINC]
    {\langle x ++, \sigma \rangle
    \Downarrow_{exp}
    \langle n + 1, \sigma \rangle}
    {\langle x, \sigma \rangle
    \Downarrow_{exp}
    \langle n, \sigma \rangle}

    
\end{adjustbox}


\section*{Ejercicio 5}

\begin{proposition}
La relación de transición es determinista: si $t \to v$ y $t \to v'$ entonces $v = v'$.
\end{proposition}

\begin{proof}
Probamos por inducción sobre la derivación $t \to v$, con análisis de casos:

\textbf{Caso ASS:}  
\[
t = \langle x = e, \sigma \rangle, \quad v = \langle \textbf{skip}, [\sigma | x : n] \rangle
\]  
Como la expresión $e$ es determinista por hipótesis inductiva, la única regla aplicable es \textsc{ASS}, luego $v = v'$.

\textbf{Caso SEQ1:}  
\[
t = \langle \textbf{skip}; c_1, \sigma \rangle, \quad v = \langle c_1, \sigma \rangle
\]  
Como esta derivación no tiene antecedente, y $t$ tiene la forma $\langle \textbf{skip}; c_1 \rangle$, la única regla posible es \textsc{SEQ1}, luego $v = v'$.

\textbf{Caso SEQ2:}  
\[
t = \langle c_0; c_1, \sigma \rangle, \quad v = \langle c_0'; c_1, \sigma' \rangle
\]  
Dado que $t$ tiene la forma $\langle c_0; c_1 \rangle$ y existe un antecedente determinista para $c_0 \to c_0'$, la única regla aplicable es \textsc{SEQ2}, luego $v = v'$.

\textbf{Caso IF1:}  
\[
t = \langle \textbf{if } b \textbf{ then } c_0 \textbf{ else } c_1, \sigma \rangle, 
\quad v = \langle c_0, \sigma' \rangle
\]  
Como $t$ comienza con un \textbf{if}, puede provenir de \textsc{IF1} o \textsc{IF2}.  
Pero como $b$ es determinista, si $b \Downarrow \textbf{true}$ no puede evaluar a \textbf{false}. Por tanto, solo aplica \textsc{IF1} y $v = v'$.

\textbf{Caso IF2:}  
\[
t = \langle \textbf{if } b \textbf{ then } c_0 \textbf{ else } c_1, \sigma \rangle, 
\quad v = \langle c_1, \sigma' \rangle
\]  
Análogo al caso anterior: como $b$ es determinista, si $b \Downarrow \textbf{false}$ no puede evaluar a \textbf{true}, luego solo aplica \textsc{IF2} y $v = v'$.

\textbf{Caso REPEAT:}  
\[
t = \langle \textbf{repeat } c \textbf{ until } b, \sigma \rangle, 
\quad v = \langle c;\ \textbf{if } b \textbf{ then skip else repeat } c \textbf{ until } b, \sigma \rangle
\]  
Como $t$ tiene la forma $\langle \textbf{repeat } c \textbf{ until } b \rangle$, la única regla aplicable es \textsc{REPEAT}, luego $v = v'$.

En todos los casos se cumple $v = v'$, por lo que la relación es determinista.
\end{proof}


\section*{Ejercicio 6}


\text{A1:}
\[
\begin{adjustbox}{center,max width=\textwidth}

  \infer[R1]
  { \langle x = x+1; y = x, \sigma \rangle 
      \leadsto^* 
      \langle \text{Skip}; y = x, [\sigma|x : x_0 + 1] \rangle }
  { 
    \infer[SEQ2]
    { \langle x = x+1; y = x, \sigma \rangle 
        \leadsto
        \langle \text{Skip}; y = x, [\sigma|x : x_0 + 1] \rangle }
    { 
      \infer[ASS]
      { \langle x = x+1, \sigma \rangle 
          \leadsto
          \langle \text{Skip}, [\sigma|x : x_0 + 1] \rangle }
      { 
        \infer[PLUS]
        { \langle x+1, \sigma \rangle 
            \Downarrow_{exp} 
            \langle x_0+1, \sigma \rangle }
        { 
          \infer[VAR]
          { \langle x, \sigma \rangle 
              \Downarrow_{exp} 
              \langle x_0, \sigma \rangle }
          { x \in dom(\sigma) } 
          &
          \infer[NVAL]
          { \langle 1, \sigma \rangle 
              \Downarrow_{exp} 
              \langle 1, \sigma \rangle }
          { }
        }
      }
    }
  }

 \end{adjustbox}
\]


\text{A2:}
\[
\begin{adjustbox}{center,max width=\textwidth}
  \infer[R1]
  { \langle \text{Skip}; y = x, [\sigma|x : x_0 + 1] \rangle
      \leadsto^* 
      \langle y = x, [\sigma|x : x_0 + 1] \rangle }
  {
    \infer[SEQ1]
    { \langle \text{Skip}; y = x, [\sigma|x : x_0 + 1] \rangle
        \leadsto
        \langle y = x, [\sigma|x : x_0 + 1] \rangle }
    { }
  }
 \end{adjustbox}
\]


\text{A3:}
\[
\begin{adjustbox}{center,max width=\textwidth}


\infer[R1]
{\langle y = x, [\sigma|x : x_0 + 1] \rangle
    \leadsto^*
    \langle \text{Skip}, [\sigma|y = x_0 + 1 \wedge x : x_0 + 1] \rangle}
{\infer[ASS]
{ \langle y = x, [\sigma|x : x_0 + 1] \rangle
    \leadsto
    \langle \text{Skip}, [\sigma|y = x_0 + 1 \wedge x : x_0 + 1] \rangle}
{\infer[VAR]
          { \langle x, [\sigma | x : x_{0} + 1] \rangle 
              \Downarrow_{exp} 
              \langle x_0 + 1 ,[\sigma | x : x_{0} + 1] \rangle }
          { x \in dom[\sigma | x : x_{0} + 1] }
          }
        }

 \end{adjustbox}
\]



\text{A4:}
\[
\begin{adjustbox}{center,max width=\textwidth}

\infer[R2]
{\langle x = x+1; y = x, \sigma \rangle 
      \leadsto^*
      \langle \text{Skip}, [\sigma|y = x_0 + 1 \wedge x : x_0 + 1] \rangle}
{
\infer[R2]
{\langle x = x+1; y = x, \sigma \rangle 
      \leadsto^* 
      \langle y = x, [\sigma|x : x_0 + 1] \rangle}
{A1 & A2}
& \qquad               A3
}

 \end{adjustbox}
\]

\[
\begin{adjustbox}{center,max width=\textwidth}



\infer[R1]
{\langle y = x++, \sigma \rangle
\leadsto^*
\langle \text{Skip}, [\sigma | y = x_0 + 1 \wedge x = x_0 + 1] \rangle}
{\infer[ASS]
{\langle y = x++, \sigma \rangle
\leadsto
\langle \text{Skip}, [\sigma | y = x_0 + 1 \wedge x = x_0 + 1] \rangle}
{\infer[VARINC]
{\langle x++, \sigma \rangle
\Downarrow_{exp}
\langle x_0 +1, [\sigma | x = x_0 + 1] \rangle}
{\infer[VAR]
{\langle x, \sigma \rangle
\Downarrow_{exp}
\langle x_0, \sigma \rangle}
{x \in Dom  (\sigma)}}}}




 \end{adjustbox}
\]


\end{document}